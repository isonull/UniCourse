\documentclass[a4paper]{article}

\def\npart {IA}
\def\nterm {Michaelmas}
\def\nyear {2016}
\def\nlecturer {xxx}
\def\ncourse {Discrete Mathematics}
\def\nauthor {Z.Yan}



\input{header}

\begin{document}
\maketitle

\begin{itemize}
\item[*]\textbf{Proof [5 lectures]}\\
Proofs in practice and mathematical jargon. Mathematical statements: implication, bi-implication, universal quantification, conjunction, existential quantification, disjunction, negation. Logical deduction: proof strategies and patterns, scratch work, logical equivalences. Proof by contradiction. Divisibility and congruences. Fermat’s Little Theorem. 

\item[*]\textbf{Numbers [5 lectures]}\\
Number systems: natural numbers, integers, rationals, modular integers. The Division Theorem and Algorithm. Modular arithmetic. Sets: membership and comprehension. The greatest common divisor, and Euclid’s Algorithm and Theorem.The Extended Euclid’s Algorithm and multiplicative inverses in modular arithmetic. The Diffie-Hellman cryptographic method. Mathematical induction: Binomial Theorem, Pascal’s Triangle, Fundamental Theorem of Arithmetic, Euclid’s infinity of primes. 

\item[*]\textbf{Sets [7 lectures]}\\
Extensionality Axiom: subsets and supersets. Separation Principle: Russell’s Paradox, the empty set. Powerset Axiom: the powerset Boolean algebra, Venn and Hasse diagrams. Pairing Axiom: singletons, ordered pairs, products. Union axiom: big unions, big intersections, disjoint unions. Relations: composition, matrices, directed graphs, preorders and partial orders. Partial and (total) functions. Bijections: sections and retractions. Equivalence relations and set partitions. Calculus of bijections: characteristic (or indicator) functions. Finite cardinality and counting. Infinity axiom. Surjections. Enumerable and countable sets. Axiom of choice. Injections. Images: direct and inverse images. Replacement Axiom: set-indexed constructions. Set cardinality: Cantor-Schoeder-Bernstein Theorem, unbounded cardinality, diagonalisation, fixed-points. Foundation Axiom. 

\item[*]\textbf{Formal languages and automata [7 lectures]}\\
Introduction to inductive definitions using rules and proof by rule induction. Abstract syntax trees. Regular expressions and their algebra. Finite automata and regular languages: Kleene’s theorem and the Pumping Lemma.


\end{itemize}

\tableofcontents

\section{Proofs}

. : implication, bi-implication, universal quantification, conjunction, existential quantification, disjunction, negation. : proof strategies and patterns, scratch work, logical equivalences. . . . 

\subsection{Proofs in practice}

\begin{itemize}
\item[]
We are interested in examining the following statement:\\
\begin{stat}
The product of two odd integers is odd.\\
\end{stat}
This seems innocuous enough, but it is in fact full of baggage.\\
For instance,it presupposes that you know:
\begin{itemize}

\item what a statement is;
\item what the integers (...,$-1$,0,1,...) are,and that amongst them there is a class of odd ones (...,$-3$,$-1$,1,3,...);
\item what the product of two integers is, and that this is in turn an integer.

\end{itemize}

\item[]
More precisely put, we may write:
\begin{stat}
If m and n are odd integers then so is $m\cdot n$.
\end{stat}
which further presupposes that you know:
\begin{itemize}
\item what variables are;
\item what $$if ... then ...$$ statements are, and how one goes about proving them;
\item that the symbol "." is commonly used to denote the product operation. 
\end{itemize}

\item[]
Even more precisely, we should write
\begin{stat}
For all integers m and n, if m and n are odd then so is $m\cdot n$.
\end{stat}
\begin{itemize}
\item what $$for\ all ...$$ statements are, and how one goes about proving them.
\end{itemize}

Thus,in trying to understand and then prove the above statement, we are assuming quite a lot of mathematical jargon that one needs to learn and practice with to make it a useful, and in fact very powerful, tool.

\end{itemize}

\subsection{Mathematical jargon}
\begin{description}
\item[Statement] A sentence that is either true or false - but not both.

\begin{eg}[1]\ 
$$e^{i\pi}+1=0$$
\end{eg}
\begin{eg}[Wrong]
This statement is false.
\end{eg}

\item[Predicate] A statement whose truth depends on the values of one or more variables.

\begin{eg}[2]\ 
\begin{enumerate}
\item $$e^{ix}=\cos x + i\sin x$$
\item the function f is differentiable
\end{enumerate}
\end{eg}

\item[Theorem] A very important true statement.
\item[Proposition] A less important but nonetheless interesting true statement.
\item[Lemma] A true statement used in proving other true statements.
\item[Corollary] A true statement that is a simple deduction from a theorem or proposition.

\begin{eg}[3]\ 
\begin{enumerate}

\item 
\begin{description}
\item[Fermat's Last Theorem] If x,y,z and n are integers satisfying 
$$x^n+y^n=z^n$$
then either $n\leq 2$ or $xyz=0$.
\end{description}

\item \begin{description}
\item[The Pumping Lemma] Let $\mathcal{L}$ be a regular language. then there is a positive integer p such that any word $w\in \mathcal{L}$ of length exceeding p can expressed as $w=xyz$, $|y|>0,|xy|\leq p$, such that, for all $i>0$, $xy^iz$ is also a word of $\mathcal{L}$.
\end{description} 
\end{enumerate}
\end{eg}

\item[Conjecture] A statement believed to be true, but for which we have no proof.

\begin{eg}[4]\ 
\begin{enumerate}
\item Goldbach's Conjecture
\item The Riemann Hypothesis
\end{enumerate}
\end{eg}

\item[Proof] Logical explanation of why a statement is true; a method for establishing truth.
\item[Logic] The study of methods and principles used to distinguish good (correct) from bad (incorrect) reasoning.

\begin{eg}[5]\ 
\begin{enumerate}
\item Classical predicate logic
\item Hoare logic
\item Temporal logic
\end{enumerate}
\end{eg}

\item[Axiom] A basic assumption about a mathematical situation.\\
Axioms cam be considered facts that do not need to be proved (just to get us going in a subject) or they can be used in definitions.
\begin{eg}[6] \ 
\begin{enumerate}
\item Euclidean Geometry
\item Riemannian Geometry
\item Hyperbolic Geometry
\end{enumerate}
\end{eg}

\item[Definition] A explanation of the mathematical meaning of a word (or phase).\\
The word (or phase) is generally defined in terms of properties.

\begin{wn}
It is vitally important that you can recall definitions precisely. A common problem is not to be able to advance in some problem because the definition of a word is unknown.
\end{wn}

\begin{defi}[7]
An integer is said to be odd whenever it is of the form $2\cdot i + 1$ for some (necessarily unique) integer i.
\end{defi}

\begin{prop}[8]
For all integers m and n, if m and n are off then so is $m\cdot n$.
\end{prop}

\begin{proof}
Let m and n be arbitrary odd integers. Thus, $m = 2\cdot i +1$ and $n = 2\cdot j +1$ for some integers i and j. Hence, we have that $m\cdot n = 2 \cdot k +1$ for $k = 2\cdot i\cdot j +i +j$, showing that $m\cdot n$ is indeed odd.
\end{proof}

\begin{wn}
Though the scratch work contains the idea behind the given proof, it is not a proper proof.
\end{wn}

\end{description}

\begin{defi}[Mathematical proof]
A mathematical proof is a sequence of logical deductions from axioms and previously proved statements that concludes with the proposition in the question.\\
The axioms-and-proof approach is called the axiomatic method. 
\end{defi}

\subsection{Mathematical statements}

\subsection{Logical deduction}
\subsection{Proof by contradiction}
\subsection{Divisibility and congruences}
\subsection{Fermat’s Little Theorem}


\section{Numbers}

\section{Sets}

\section{Regular languages and finite automata}

\end{document}