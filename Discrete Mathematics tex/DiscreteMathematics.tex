\documentclass[a4paper]{article}

\def\npart {IA}
\def\nterm {Michaelmas}
\def\nyear {2016}
\def\nlecturer {xxx}
\def\ncourse {Discrete Mathematics}
\def\nauthor {Z.Yan}

\input{header}

\begin{document}
\maketitle

\begin{itemize}
\item[*]\textbf{Proof [5 lectures]}\\
Proofs in practice and mathematical jargon. Mathematical statements: implication, bi-implication, universal quantification, conjunction, existential quantification, disjunction, negation. Logical deduction: proof strategies and patterns, scratch work, logical equivalences. Proof by contradiction. Divisibility and congruences. Fermat’s Little Theorem. 

\item[*]\textbf{Numbers [5 lectures]}\\
Number systems: natural numbers, integers, rationals, modular integers. The Division Theorem and Algorithm. Modular arithmetic. Sets: membership and comprehension. The greatest common divisor, and Euclid’s Algorithm and Theorem.The Extended Euclid’s Algorithm and multiplicative inverses in modular arithmetic. The Diffie-Hellman cryptographic method. Mathematical induction: Binomial Theorem, Pascal’s Triangle, Fundamental Theorem of Arithmetic, Euclid’s infinity of primes. 

\item[*]\textbf{Sets [7 lectures]}\\
Extensionality Axiom: subsets and supersets. Separation Principle: Russell’s Paradox, the empty set. Powerset Axiom: the powerset Boolean algebra, Venn and Hasse diagrams. Pairing Axiom: singletons, ordered pairs, products. Union axiom: big unions, big intersections, disjoint unions. Relations: composition, matrices, directed graphs, preorders and partial orders. Partial and (total) functions. Bijections: sections and retractions. Equivalence relations and set partitions. Calculus of bijections: characteristic (or indicator) functions. Finite cardinality and counting. Infinity axiom. Surjections. Enumerable and countable sets. Axiom of choice. Injections. Images: direct and inverse images. Replacement Axiom: set-indexed constructions. Set cardinality: Cantor-Schoeder-Bernstein Theorem, unbounded cardinality, diagonalisation, fixed-points. Foundation Axiom. 

\item[*]\textbf{Formal languages and automata [7 lectures]}\\
Introduction to inductive definitions using rules and proof by rule induction. Abstract syntax trees. Regular expressions and their algebra. Finite automata and regular languages: Kleene’s theorem and the Pumping Lemma.


\end{itemize}

\tableofcontents

\section{Proofs}

. : implication, bi-implication, universal quantification, conjunction, existential quantification, disjunction, negation. : proof strategies and patterns, scratch work, logical equivalences. . . . 

\subsection{Proofs in practice and mathematical jargon}

\begin{itemize}
\item[]
We are interested in examining the following statement:\\
\begin{stat}
The product of two odd integers is odd.\\
\end{stat}
This seems innocuous enough, but it is in fact full of baggage.\\
For instance,it presupposes that you know:
\begin{itemize}

\item what a statement is;
\item what the integers (...,$-1$,0,1,...) are,and that amongst them there is a class of odd ones (...,$-3$,$-1$,1,3,...);
\item what the product of two integers is, and that this is in turn an integer.

\end{itemize}



\end{itemize}
\subsection{Mathematical statements}
\subsection{Logical deduction}
\subsection{Proof by contradiction}
\subsection{Divisibility and congruences}
\subsection{Fermat’s Little Theorem}


\section{Numbers}

\section{Sets}

\section{Regular languages and finite automata}

\end{document}