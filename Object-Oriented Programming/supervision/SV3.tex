\documentclass[10pt,twoside,a4paper]{article}

% Configure these parameters.
% Name and email
\newcommand{\studentname}{Joe Yan}
\newcommand{\studentemail}{zy275@cam.ac.uk}

% Date work done
\newcommand{\svworkdate}{2016-11-21}

% Details of supervision
\newcommand{\svcourse}{CST Part IA: OOP}
\newcommand{\svnumber}{3}
\newcommand{\svdate}{2016-11-26}
\newcommand{\svtime}{19:00}
\newcommand{\svvenue}{Churchill 1C}
\newcommand{\svrname}{Dr John Fawcett}
\newcommand{\svrinit}{JKF}
% End configuration

\usepackage{a4}             % Adjust margins for A4 media
\usepackage{pgfplots}
\usepackage{fancyhdr}
\renewcommand{\headrulewidth}{0.4pt}
\renewcommand{\footrulewidth}{0.4pt}
\fancyheadoffset[LO,LE,RO,RE]{0pt}
\fancyfootoffset[LO,LE,RO,RE]{0pt}
\pagestyle{fancy}
\fancyhead{}
\fancyhead[LO,RE]{{\bfseries \studentname}\\\studentemail}
\fancyhead[RO,LE]{{\bfseries \svcourse, SV~\svnumber}\\\svdate\ \svtime, \svvenue}
\fancyfoot{}
\fancyfoot[LO,RE]{For: \svrname}
\fancyfoot[RO,LE]{\thepage\ / \pageref{LastPage}}
\fancyfoot[C]{\today}

\usepackage{lastpage}       % "n of m" page numbering
\usepackage{lscape}         % Makes landscape easier
%\usepackage{portland}      % Switch between portrait and landscape
\usepackage{graphics}       % Graphics commands
\usepackage{wrapfig}        % Wrapping text around figures
\usepackage{epsfig}         % Embed encapsulated postscript
\usepackage{rotating}       % Extra graphics rotation
%\usepackage{tables}        % Tabular environments
\usepackage{longtable}      % Page breaks within tables
\usepackage{supertabular}   % Page breaks within tables
\usepackage{multicol}       % Allows table cells to span cols
\usepackage{multirow}       % Allows table cells to span rows
\usepackage{texnames}       % Macros for common tex names
%\usepackage{trees}         % Tree-like layout
\usepackage{hhline}         % Horizontal lines in tables
\usepackage{siunitx}        % Correct spacing of units

\usepackage{listings}       % Source code listings
\usepackage{array}          % Array environment
\usepackage{hyperref}       % URL formatting
\usepackage{amsmath}        % American Mathematical Society
\usepackage{amssymb}        % Maths symbols
\usepackage{amsthm}         % Theorems
%\usepackage{mathpartir}    % Proofs and inference rules
\usepackage{verbatim}       % Verbatim blocks
\usepackage{ifthen}         % Conditional processing in tex
\usepackage{xcolor}         % X11 colour names

% control width and vertically align text in table cells
\newcolumntype{L}[1]{>{\raggedright\let\newline\\\arraybackslash\hspace{0pt}}m{#1}}
\newcolumntype{C}[1]{>{\centering\let\newline\\\arraybackslash\hspace{0pt}}m{#1}}
\newcolumntype{R}[1]{>{\raggedleft\let\newline\\\arraybackslash\hspace{0pt}}m{#1}}

% make hyperref links not-ugly
\hypersetup{
    colorlinks=false,
    pdfborder={0 0 0},
}

\renewcommand{\oddsidemargin}{-20pt}
\renewcommand{\evensidemargin}{-20pt}
\renewcommand{\topmargin}{-30pt}
\renewcommand{\textwidth}{410pt}
\renewcommand{\marginparwidth}{100pt}

\setlength{\parindent}{0em}
\addtolength{\parskip}{1ex}

\usepackage[draft]{changes}
\setauthormarkup[left]{\textbf{[#1]}~}
\definechangesauthor[\svrname]{\svrinit}{orange}
\newcommand{\jkfadd}[1]{\added[\svrinit]{#1}}
\newcommand{\jkfdel}[1]{\deleted[\svrinit]{#1}}
\newcommand{\jkfrep}[2]{\replaced[\svrinit]{#1}{#2}}
\newcommand{\jkfmar}[1]{\marginpar{\jkfadd{#1}}}

\begin{document}

\author{\studentname}
\title{\svcourse, SV~\svnumber}
\date{\svworkdate}

\textbf{\svcourse, SV~\svnumber}\\
\textbf{\studentname}\\
\textbf{\svworkdate}\\

\section{Exercise 40}

\begin{lstlisting}
public class MarksBoard {
	private Map<String, Float> mMapName;
	// hash fast to add new element but slow when doing sorting.
	// tree slow to add new element but fast to request sorted list.

	public MarksBoard() {
		mMapName = new HashMap<>();
	}

	public boolean addStudent(String name, float score) {
		if (mMapName.containsKey(name))
			return false;
		else {
			mMapName.put(name, score);
			return true;
		}
	}

	public List<String> allStudents() {
		List<String> students = new ArrayList<String>(mMapName.keySet());
		Collections.sort(students);
		return students;
	}

	public float getPercentageBoundary(float p) {
		if (p <= 0f)
			return 100;
		if (p > 100f)
			return 0;
		float size = mMapName.size();
		int nth = (int) Math.ceil(size * p / 100);
		ArrayList<Float> scoreList = new ArrayList<>(mMapName.values());
		Collections.sort(scoreList);
		return scoreList.get((int) size - nth);
	}

	public List<String> getTopPercentageStudents(float p) {
		float n = mMapName.size();
		float boundary = getPercentageBoundary(p);
		ArrayList<String> nameList = new ArrayList<>();
		for (java.util.Map.Entry<String, Float> pair : mMapName.entrySet()) {
			if (pair.getValue() >= boundary)
				nameList.add(pair.getKey());
		}
		Collections.sort(nameList);
		return nameList;
	}

	public float getMedianMark() {
		int size = mMapName.size();
		int pos = size / 2;
		ArrayList<Float> scoreList = new ArrayList<>(mMapName.values());
		Collections.sort(scoreList);
		if (size % 2 == 1) {
			return scoreList.get(pos);
		} else {
			return (scoreList.get(pos) + scoreList.get(pos - 1)) / 2;
		}
	}
}
\end{lstlisting}

\section{Exercise 47}

\begin{lstlisting}
public class IntPair implements Comparable<IntPair> {
	private int first;
	private int second;

	public IntPair(int a, int b) {
		first = a;
		second = b;
	}

	@Override
	public int compareTo(IntPair o) {
		if (this.first > o.first) {
			return 1;
		} else if (this.first < o.first) {
			return -1;
		} else {
			if (this.second > o.second) {
				return 1;
			} else if (this.second < o.second) {
				return -1;
			} else {
				return 0;
			}
		}
	}

	public int first() {
		return first;
	}

	public int second() {
		return second;
	}
}

public class ReadAndSort {
	public static void main(String args[]) throws Exception {
		try {
			Reader r = new FileReader
			("D:/Java_Workspace/OOPSV/src/S47B/test.txt");
			BufferedReader b = new BufferedReader(r);
			String firstLine = b.readLine();
			String secondLine = b.readLine();
			String[] firsts = firstLine.split(",");
			String[] seconds = secondLine.split(",");
			if (firsts.length != seconds.length)
				throw new Exception("The data is not in pair!");
			List<IntPair> data = new LinkedList<IntPair>();
			for (int i = 0; i != firsts.length; ++i) {
				data.add(new IntPair(Integer.parseInt(firsts[i]),
				Integer.parseInt(seconds[i])));
			}
			Collections.sort(data);
			for (IntPair pair : data) {
				System.out.print(pair.first() + "  ");
				System.out.println(pair.second());
			}
			b.close();
		} catch (FileNotFoundException e) {
			System.out.println("File not found!");
			e.printStackTrace();
		} catch (IOException e) {
			System.out.println("Exception happens when reading file.");
			e.printStackTrace();
		} catch (NumberFormatException e) {
			System.out.println("Parsing to int failed.");
		}
	}
}
\end{lstlisting}
\newpage
\section{Exercise 49}
\begin{itemize}
\item The main purpose of the state pattern design is for several states which extends or implements the same base class are able to switch to each other swiftly. If I use a abstract base class and subclasses simply inherit from it the design will lose the flexibility in the state changing, I have to write something like 
\begin{lstlisting}
Academic someone = new lecturer();
Academic someone = (professor) someone; // I guess complier will accept this?
\end{lstlisting}
which is clumsy.
\item Also the trivial design is bad for encapsulation. Instead of using a context class accepting the request of function and change of state from package user, a abstract base class will explode the inner design to the user which is not expected. 
\end{itemize}

\end{document}

