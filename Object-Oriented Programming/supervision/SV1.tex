\documentclass[10pt,twoside,a4paper]{article}

% Configure these parameters.
% Name and email
\newcommand{\studentname}{Joe Yan}
\newcommand{\studentemail}{zy275@cam.ac.uk}

% Date work done
\newcommand{\svworkdate}{2016-11-6}

% Details of supervision
\newcommand{\svcourse}{CST Part IA: OOP}
\newcommand{\svnumber}{1}
\newcommand{\svdate}{2016-11-12}
\newcommand{\svtime}{19:00}
\newcommand{\svvenue}{Churchill 1C}
\newcommand{\svrname}{Dr John Fawcett}
\newcommand{\svrinit}{JKF}
% End configuration

\usepackage{a4}             % Adjust margins for A4 media
\usepackage{pgfplots}
\usepackage{fancyhdr}
\renewcommand{\headrulewidth}{0.4pt}
\renewcommand{\footrulewidth}{0.4pt}
\fancyheadoffset[LO,LE,RO,RE]{0pt}
\fancyfootoffset[LO,LE,RO,RE]{0pt}
\pagestyle{fancy}
\fancyhead{}
\fancyhead[LO,RE]{{\bfseries \studentname}\\\studentemail}
\fancyhead[RO,LE]{{\bfseries \svcourse, SV~\svnumber}\\\svdate\ \svtime, \svvenue}
\fancyfoot{}
\fancyfoot[LO,RE]{For: \svrname}
\fancyfoot[RO,LE]{\thepage\ / \pageref{LastPage}}
\fancyfoot[C]{\today}

\usepackage{lastpage}       % "n of m" page numbering
\usepackage{lscape}         % Makes landscape easier
%\usepackage{portland}      % Switch between portrait and landscape
\usepackage{graphics}       % Graphics commands
\usepackage{wrapfig}        % Wrapping text around figures
\usepackage{epsfig}         % Embed encapsulated postscript
\usepackage{rotating}       % Extra graphics rotation
%\usepackage{tables}        % Tabular environments
\usepackage{longtable}      % Page breaks within tables
\usepackage{supertabular}   % Page breaks within tables
\usepackage{multicol}       % Allows table cells to span cols
\usepackage{multirow}       % Allows table cells to span rows
\usepackage{texnames}       % Macros for common tex names
%\usepackage{trees}         % Tree-like layout
\usepackage{hhline}         % Horizontal lines in tables
\usepackage{siunitx}        % Correct spacing of units

\usepackage{listings}       % Source code listings
\usepackage{array}          % Array environment
\usepackage{hyperref}       % URL formatting
\usepackage{amsmath}        % American Mathematical Society
\usepackage{amssymb}        % Maths symbols
\usepackage{amsthm}         % Theorems
%\usepackage{mathpartir}    % Proofs and inference rules
\usepackage{verbatim}       % Verbatim blocks
\usepackage{ifthen}         % Conditional processing in tex
\usepackage{xcolor}         % X11 colour names

% control width and vertically align text in table cells
\newcolumntype{L}[1]{>{\raggedright\let\newline\\\arraybackslash\hspace{0pt}}m{#1}}
\newcolumntype{C}[1]{>{\centering\let\newline\\\arraybackslash\hspace{0pt}}m{#1}}
\newcolumntype{R}[1]{>{\raggedleft\let\newline\\\arraybackslash\hspace{0pt}}m{#1}}

% make hyperref links not-ugly
\hypersetup{
    colorlinks=false,
    pdfborder={0 0 0},
}

\renewcommand{\oddsidemargin}{-20pt}
\renewcommand{\evensidemargin}{-20pt}
\renewcommand{\topmargin}{-30pt}
\renewcommand{\textwidth}{410pt}
\renewcommand{\marginparwidth}{100pt}

\setlength{\parindent}{0em}
\addtolength{\parskip}{1ex}

\usepackage[draft]{changes}
\setauthormarkup[left]{\textbf{[#1]}~}
\definechangesauthor[\svrname]{\svrinit}{orange}
\newcommand{\jkfadd}[1]{\added[\svrinit]{#1}}
\newcommand{\jkfdel}[1]{\deleted[\svrinit]{#1}}
\newcommand{\jkfrep}[2]{\replaced[\svrinit]{#1}{#2}}
\newcommand{\jkfmar}[1]{\marginpar{\jkfadd{#1}}}

\begin{document}

\author{\studentname}
\title{\svcourse, SV~\svnumber}
\date{\svworkdate}

\textbf{\svcourse, SV~\svnumber}\\
\textbf{\studentname}\\
\textbf{\svworkdate}\\

\section{Example Sheet 1314 3B}
\begin{lstlisting}
public static float[][] creatMatrix(int n) {
	float[][] mat = new float[n][n];
	return mat;
}
public static float[][] transpose(float[][] mat) {
	float store;
	int len = mat.length;	for(int i = 0; i < len; ++i) {
		for(int j = i; j < len; ++j) {
			store = mat[i][j];
			mat[i][j] = mat[j][i];
			mat[j][i] = store;
		}
	}
	return mat;
}
\end{lstlisting}
\section{Example Sheet 1314 6A}
\includegraphics[scale=0.1]{sv1-2.png}

\section{Example Sheet 1314 11B}
\begin{enumerate}

\item
\begin{lstlisting}
public class Vector2D {
	public float a;
	public float b;
	
	public Vector2D() {
		a=0;
		b=0;
	}
	public Vector2D(float x,float y) {
		a = x;
		b = y;
	}

	public static float scalarProduct(Vector2D x, Vector2D y) {
		float t = x.a*y.a+x.b*y.b;
		return t;
	}
	
	public static Vector2D normalise(Vector2D x) {
		float mag = magnitude(x);
		return new Vector2D(x.a/mag,x.b/mag);
	}
	
	public static float magnitude(Vector2D x) {
		float t = (float) Math.sqrt(x.a*x.a+x.b*x.b);
		return t;
	}
	
	public void add(Vector2D v) {
		this.a = this.a + v.a;
		this.b = this.b + v.b;
	}
}
\end{lstlisting}
\item
\begin{itemize}
\item Make fields private
\item Be careful when the reference is passed in or out, when construct the object or provide the specific field, always make a new reference for it. (Not for this example)
\item Final the class or method preventing it from overriding
\item Do not provide any methods can mutate the field
\end{itemize}
\item
Mutable:\\
\begin{lstlisting}
public void add(Vector2D v) {
		this.a = this.a + v.a;
		this.b = this.b + v.b;
	}
\end{lstlisting}
Immutable:\\
\begin{lstlisting}	

public Vector2D add1(Vector2D v) {
	return new Vector2D(this.a+v.a,this.b+v.b);
}
public Vector2D add(Vector2D v1, Vector2D v2) {
	return new Vector2D(v1.a+v2.a,v1.b+v2.b);
}
public static Vector2D add1(Vector2D v1, Vector2D v2) {
	return new Vector2D(v1.a+v2.a,v1.b+v2.b);
}
\end{lstlisting}
\item
Declare in annotation?
\end{enumerate}
I do not quite understand what is 3 and 4 asking for..

\section{Example Sheet 1314 12B}

\begin{lstlisting}
public class OOPLinkedList {
	protected OOPLinkedListElement hd;
	
	public OOPLinkedList() {
		hd = null;
	}
	
	public void add(int x) {
		hd = new OOPLinkedListElement(x,hd);
	}
	public void remove() {
		if(hd == null) hd = null;
		else hd = hd.getnext();
	}
	public int get() throws Exception {
		if(hd==null)throw new Exception("List is null.");
		else return hd.getele();
	}
	
	public int length() {
		if(hd == null) return 0;
		int len = 1;
		OOPLinkedListElement p = hd;
		while(p.getnext() != null){
			++len;
			p=p.getnext();
		}
		return len;
	}
}

public class OOPLinkedListElement {
	private int ele;
	protected OOPLinkedListElement next;
	
	public OOPLinkedListElement(int x) {
		ele = x;
		next = null;
	}
	
	public OOPLinkedListElement(int x,OOPLinkedListElement xs) {
		ele = x;
		next = xs;
	}
	
	public int getele() {
		return ele;
	}
	
	public OOPLinkedListElement getnext() {
		return next;
	}
}
\end{lstlisting}

\section{Example Sheet 1314 18B}
\begin{lstlisting}

public class OOPSortedLinkedList extends OOPLinkedList{

	public OOPSortedLinkedList() {
		super();
	}
	
	public void add(int x) {
		if(hd == null) {
			hd = new OOPLinkedListElement(x,null);
			return;
		}
		if(x<hd.getele()) {
			hd = new OOPLinkedListElement(x,hd.getnext());
			return;
		}
		OOPLinkedListElement before = hd;
		while(before.getnext() != null && x > before.getnext().getele()) {
			before = before.getnext();
		}
		before.next = new OOPLinkedListElement(x,before.getnext());
	}
}
\end{lstlisting}

\end{document}

