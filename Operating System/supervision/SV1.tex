\documentclass[10pt,twoside,a4paper]{article}

% Configure these parameters.
% Name and email
\newcommand{\studentname}{Joe Yan}
\newcommand{\studentemail}{zy275@cam.ac.uk}

% Date work done
\newcommand{\svworkdate}{2017-2-17}

% Details of supervision
\newcommand{\svcourse}{CST Part IA: Operating System}
\newcommand{\svnumber}{1}
\newcommand{\svdate}{2016-2-25}
\newcommand{\svtime}{1700}
\newcommand{\svvenue}{Churchill College}
\newcommand{\svrname}{Lucas Sonnabend}
\newcommand{\svrinit}{JKF}
% End configuration

\usepackage{a4}             % Adjust margins for A4 media
\usepackage{pgfplots}
\usepackage{fancyhdr}
\renewcommand{\headrulewidth}{0.4pt}
\renewcommand{\footrulewidth}{0.4pt}
\fancyheadoffset[LO,LE,RO,RE]{0pt}
\fancyfootoffset[LO,LE,RO,RE]{0pt}
\pagestyle{fancy}
\fancyhead{}
\fancyhead[LO,RE]{{\bfseries \studentname}\\\studentemail}
\fancyhead[RO,LE]{{\bfseries \svcourse, SV~\svnumber}\\\svdate\ \svtime, \svvenue}
\fancyfoot{}
\fancyfoot[LO,RE]{For: \svrname}
\fancyfoot[RO,LE]{\thepage\ / \pageref{LastPage}}
\fancyfoot[C]{\today}

\usepackage{lastpage}       % "n of m" page numbering
\usepackage{lscape}         % Makes landscape easier
%\usepackage{portland}      % Switch between portrait and landscape
\usepackage{graphics}       % Graphics commands
\usepackage{wrapfig}        % Wrapping text around figures
\usepackage{epsfig}         % Embed encapsulated postscript
\usepackage{rotating}       % Extra graphics rotation
%\usepackage{tables}        % Tabular environments
\usepackage{longtable}      % Page breaks within tables
\usepackage{supertabular}   % Page breaks within tables
\usepackage{multicol}       % Allows table cells to span cols
\usepackage{multirow}       % Allows table cells to span rows
\usepackage{texnames}       % Macros for common tex names
%\usepackage{trees}         % Tree-like layout
\usepackage{hhline}         % Horizontal lines in tables
\usepackage{siunitx}        % Correct spacing of units

\usepackage{listings}       % Source code listings
\usepackage{array}          % Array environment
\usepackage{hyperref}       % URL formatting
\usepackage{amsmath}        % American Mathematical Society
\usepackage{amssymb}        % Maths symbols
\usepackage{amsthm}         % Theorems
%\usepackage{mathpartir}    % Proofs and inference rules
\usepackage{verbatim}       % Verbatim blocks
\usepackage{ifthen}         % Conditional processing in tex
\usepackage{xcolor}         % X11 colour names

% control width and vertically align text in table cells
\newcolumntype{L}[1]{>{\raggedright\let\newline\\\arraybackslash\hspace{0pt}}m{#1}}
\newcolumntype{C}[1]{>{\centering\let\newline\\\arraybackslash\hspace{0pt}}m{#1}}
\newcolumntype{R}[1]{>{\raggedleft\let\newline\\\arraybackslash\hspace{0pt}}m{#1}}

% make hyperref links not-ugly
\hypersetup{
    colorlinks=false,
    pdfborder={0 0 0},
}

\renewcommand{\oddsidemargin}{-20pt}
\renewcommand{\evensidemargin}{-20pt}
\renewcommand{\topmargin}{-30pt}
\renewcommand{\textwidth}{410pt}
\renewcommand{\marginparwidth}{100pt}

\setlength{\parindent}{0em}
\addtolength{\parskip}{1ex}

\usepackage[draft]{changes}
\setauthormarkup[left]{\textbf{[#1]}~}
\definechangesauthor[\svrname]{\svrinit}{orange}
\newcommand{\jkfadd}[1]{\added[\svrinit]{#1}}
\newcommand{\jkfdel}[1]{\deleted[\svrinit]{#1}}
\newcommand{\jkfrep}[2]{\replaced[\svrinit]{#1}{#2}}
\newcommand{\jkfmar}[1]{\marginpar{\jkfadd{#1}}}

\begin{document}

\author{\studentname}
\title{\svcourse, SV~\svnumber}
\date{\svworkdate}

\textbf{\svcourse, SV~\svnumber}\\
\textbf{\studentname}\\
\textbf{\svworkdate}\\

\section{2008P1Q2}
\begin{itemize}
\item[(a)]
\begin{itemize}
\item[(i)] $2^{15}+2^{12}+2^{10}+2^8=38144$
\item[(ii)] $-(2^{12}+2^{10}+2^8)=-5376$
\item[(iii)] $- Inverse((1001 0101 0000 0000)_2 -1)=-(0110 1011 0000 0000)_2=-27392$
\end{itemize}
\item[(b)]
$C=1$ Because $N+N>2^{16}-1$
\\$V=1$ Because if we treat $N$ as a signed value, add N to N will change the sign. (i.e. Adding two negative number gives a positive number.)
\end{itemize}

\section{2011P2Q3}
\begin{itemize}
\item[(a)]
\begin{itemize}
\item[(i)] Access Control means the restriction to subjects of access to objects.
\item[(ii)] Access Control List is storing a list of subjects and rights with each objects.
\item[(iii)] Capacities are storing a list of objects and rights with each subjects.
\end{itemize}
\end{itemize}


\section{2008P1Q8}
\begin{itemize}
\item[(a)]
There are 5 process states: 1.Creation, 2.Ready, 3.Running, 4.Blocked, 5.Termination.
\\Relation:1-2,2-3,3-2,3-4,3-5,4-2.
\begin{itemize}
\item[Creation]
\begin{itemize}
\item System initialization.
\item Execution of a process creation call by a running process.
\item A subject(user)'s request to create a new process.
\item Initiation of a batch job. (Automatically following another job)
\end{itemize}
\item[Ready]
\begin{itemize}
\item After creation or taken some events ready to run in CPU but waiting for other processes.
\item Current process run for a long time, OS pause it for a while leave CPU for other process.
\end{itemize}
\item[Running]
\begin{itemize}
\item Using the CPU at that instant.
\end{itemize}
\item[Blocked]
\begin{itemize}
\item Unable to run until some external event happens.
\end{itemize}
\item[Termination]
\begin{itemize}
\item Finished execution, normal exit.
\item Error exit. (e.g. unhandled exception)
\item Fatal exit. (e.g. try to access privileged unit or fetch memory without rights)
\item Killed by other process.
\end{itemize}
\end{itemize}
\end{itemize}

\section{2007P1Q7}
\begin{itemize}
\item[(a)] Note $1024*1024*4096=2^{32}$.
\\So there is a bijection between set of all 32-bit virtual address and set of all bytes in the memory.
\\From the most to the least significant bit in the virtual address, assume following virtual memory address structure.
\\P1 = first 10 bits. P2 = following 10 bits. Offset = the last 12 bits.
\\The MMU first uses P1 to index into the first-level page table and obtain entry P1 which corresponding to a second-level page table. The MMU then use P2 to index into this second-level page table and obtain entry P2 which should be corresponding to some memory chunk with length 4096 bytes. If the Present/Absent bit is false then this is causing a page fault. Otherwise the page is in the memory, the page frame number is taken from the second-level page table is combined with Offset to construct the physical address which will be put on bus and send to memory.
\item[(b)]
-
\begin{itemize}
\item[Present bit] Whether the entry is a map to a valid physical address.
\item[Read bit] Whether the process has the right to read the physical address mapped to.
\item[Write bit]Whether the process has the right to write the physical address mapped to.
\item[Execute bit]Whether the process has the right to execute the physical address mapped to.
\end{itemize}
\item[(c)]
The paging address is one dimensional which goes from 0 to some maximum address.
\\The segmentation can have two or more segments which have separated virtual address spaces.
\\This can be implemented by restructuring the virtual address.
\\Segment number - Page number - Offset
\\The MMU will first map segment number to some page table (segment).
\\Then it maps the page number to a physical address in that page table.
\\After checking protection bits, the physical address will be sent to memory on the bus.
\end{itemize}
\end{document}